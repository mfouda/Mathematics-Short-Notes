\documentclass[10pt]{article}
\usepackage[utf8]{inputenc}
\usepackage[OT1]{fontenc}
\usepackage{amsfonts, amsmath, amsthm, amssymb}
\usepackage{bbm}
\usepackage{mathtools}
\usepackage{natbib}
\usepackage{graphicx}
\usepackage{listings}
\usepackage[margin=1in]{geometry}
\usepackage{xcolor}
\usepackage{bigints}
\usepackage{glossaries}
\usepackage{graphicx}
\usepackage{enumerate}

\theoremstyle{definition}
\newtheorem{definition}{Definition}[section]

\newtheorem{theorem}{Theorem}
\newtheorem{proposition}{Proposition}
\newtheorem{example}{Example}

\newcounter{countCode}
\lstnewenvironment{code} [1][caption=Ponme caption, label=default]{%
	\renewcommand*{\lstlistingname}{Listado} 
	\setcounter{lstlisting}{\value{countCode}} 
	\lstset{ %
	language=java,
	basicstyle=\ttfamily\footnotesize,       % the size of the fonts that are used for the code
	numbers=left,                   % where to put the line-numbers
	numberstyle=\sc,      % the size of the fonts that are used for the line-numbers
	stepnumber=1,                   % the step between two line-numbers. 
	numbersep=5pt,                 % how far the line-numbers are from the code
	numberstyle=\color{red!50!blue},
    	backgroundcolor=\color{lightgray!20},
	rulecolor=\color{blue},
	keywordstyle=\color{red}\bfseries,
	showspaces=false,               % show spaces adding particular underscores
	showstringspaces=false,         % underline spaces within strings
	showtabs=false,                 % show tabs within strings adding particular underscores
	frame=single,                   % adds a frame around the code
	framexleftmargin=0mm,
	numberblanklines=false,
	xleftmargin=5pt,
	breaklines=true,
	breakatwhitespace=true,
	breakautoindent=true,
	captionpos=t,
	texcl=true,
	tabsize=2,                      % sets default tabsize to 3 spaces
	extendedchars=true,
	inputencoding=utf8, 
	escapechar=\%,
	morekeywords={print, println, size, background, strokeWeight, fill, line, rect, ellipse, triangle, arc, save, PI, HALF_PI, QUARTER_PI, TAU, TWO_PI, width, height,},
	emph=[1]{print,println,}, emphstyle=[1]{\color{blue}}, % Mis palabras clave.
	emph=[2]{width,height,}, emphstyle=[2]{\bf\color{violet}}, % Mis palabras clave.
	emph=[3]{PI, HALF_PI, QUARTER_PI, TAU, TWO_PI}, emphstyle=[3]\color{orange!50!violet}, % Mis palabras clave.
	emph=[4]{line, rect, ellipse, triangle, arc,}, emphstyle=[4]\color{green!70!black}, % Mis palabras clave.
	%emph=[5]{size, background, strokeWeight, fill,}, emphstyle=[5]{\tt \color{red!30!blue}}, % Mis palabras clave.
	%emph={[2]sqrt,baset}, emphstyle={[2]\color{blue}}, % f(sqrt(2)), sqrt a nivel 2 se pondrá azul
	#1}}{\addtocounter{countCode}{1}}



\title{Lista de Exercícios 2}
\author{}
\date{\today}
\begin{document}
\maketitle

\noindent
\textbf{Questão 1.}
Prove que não existe função contínua $f:[a,b] \to \mathbb R$
que assuma cada um dos seus valores $f(x)$ exatamente duas vezes.
\vspace{5mm}

\noindent
\textbf{Questão 2.}
Se toda função contínua $f: X \to \mathbb R$ é uniformemente contínua, prove que o
conjunto $X$ é fechado, porém não necessariamente compacto.
\vspace{5mm}

\noindent
\textbf{Questão 3.}
Seja $f:\mathbb R \to \mathbb R$ contínua. Se existem
$\lim_{x\to +\infty} f(x)$ e
$\lim_{x\to -\infty} f(x)$, prove que $f$ é uniformemente contínuna.
\vspace{5mm}

\noindent
\textbf{Questão 4.}
Seja $f:\mathbb R \to \mathbb R$ derivável em $a$. Sejam $(x_n)$ e $(y_n)$ sequências
de pontos tais que $\lim_{n\to \infty} x_n = \lim_{n\to \infty} y_n = a$
e $x_n < a < y_n$. Prove que:
\begin{equation*}
	\lim_{n\to \infty } \frac{f(x_n)-f(y_n)}{x_n - y_n} = f'(a)
\end{equation*}
\vspace{5mm}

\noindent
\textbf{Questão 5.}
Prove que $f$ é convexa no intervalo $I$ se, e somente se, para todos $x,y \in I$, tem-se
que a função $g(t):= f(y + t(x-y))$ é convexa em $[0,1]$.
\vspace{5mm}

\noindent
\textbf{Questão 6.}
Seja $f$ contínua em $[a,b]$ e duas vezes derivável em $(a,b)$. Sejam
$A:=(a,f(a))$ e $B:=(b,f(b))$. Suponha que se o segmento de reta ligando
$A$ e $B$ intersecta o gráfico da $f$ num terceiro ponto (diferente de $A$ e $B$),
então existe $c\in (a,b)$ tal que $f''(c)=0$
\vspace{5mm}

\noindent
\textbf{Questão 7.}
Seja $f$ contínua em $(a,b)$, e derivável em $(a,b)$, exceto talvez em $c \in (a,b)$.
Prove que se $\lim_{x \to c} f'(x)$ existe e é finito igual a $A$, então $f$
deve ser derivável em $c$ e $f'(c) = A$.
\vspace{5mm}
\nocite{*}

  \bibliography{ref}
  \bibliographystyle{plainnat}
\end{document}

