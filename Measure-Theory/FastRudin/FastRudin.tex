\documentclass[10pt]{article}
\usepackage[utf8]{inputenc}
\usepackage[OT1]{fontenc}
\usepackage{amsfonts, amsmath, amsthm, amssymb}
\usepackage{bbm}
\usepackage{mathtools}
\usepackage{natbib}
\usepackage{graphicx}
\usepackage{listings}
\usepackage[margin=1in]{geometry}
\usepackage{xcolor}
\usepackage{bigints}
\usepackage{glossaries}
\usepackage{graphicx}
\usepackage{enumerate}

\theoremstyle{definition}
\newtheorem{definition}{Definition}[section]

\newtheorem{theorem}{Theorem}
\newtheorem{proposition}{Proposition}
\newtheorem{example}{Example}

\title{Fast Rudin}
\author{Davi Sales Barreira}
\date{\today}
\begin{document}
\maketitle
\begin{abstract}
	This document are condensed notes on the book
	``Real and Complex Analysis'' \citep{rudin1987real}. The goal is to cover the
	first 9 chapters of the book.
\end{abstract}

\section{Abstract Integration}

\begin{definition}[Topological Space]
	$(X,\tau)$ is a topological space where $\tau$ is the collection of open
	sets in $X$, such that:
	\begin{enumerate}[(i)]
		\item $X \in \tau$;
		\item Se $A_1,...,A_n \in \tau \implies \cap_{i=1}^n A_i \in \tau$;
		\item Se $A_\alpha \in \tau$ for any $\alpha \in \Lambda \implies \cup_{\alpha \in \Lambda} A_\alpha \in \tau$;
	\end{enumerate}
	$\tau$ is the topology of $X$.
\end{definition}

\begin{definition}[Measurable Space]
	$(X,\mathcal F)$ is a measurable space where $\mathcal F$ is a $\sigma$-algebra in $X$.
	A $\sigma$-algebra is defined such that:
	\begin{enumerate}[(i)]
		\item $X \in \mathcal F$;
		\item If $A \in \mathcal F$ then $A^c \in \mathcal F$;
		\item If $A_n \in \mathcal F \ \forall n \in \mathbb N$ then $\cup_{n \in \mathbb N} A_n \in \mathcal F$;
	\end{enumerate}
\end{definition}

\begin{definition}[Continuous Function]
	For $(X,\tau)$ and $(Y,\tau')$ topological spaces, we say $f:X \to Y$ is continuous if for every open set
	$V \subset Y$ we have that $f^{-1}(V) \subset X$ is open.
\end{definition}

\begin{definition}[Measurable Function]
	For $(X,\mathcal F)$ a measurable space and $(Y,\tau')$ a topological space, we say $f:X \to Y$ is
	$\mathcal F$-measurable if for every open set
	$V \subset Y$ we have that $f^{-1}(V) \in \mathcal F$.
\end{definition}

  \bibliography{ref}
  \bibliographystyle{plainnat}
\end{document}

